\chapter{Conclusion}
\label{Chapter6}
\lhead{Chapter 6. \emph{Conclusion}}

This research has successfully developed and validated an enhanced machine learning framework for Primary User Emulation Attack (PUEA) detection in Cognitive Radio Networks, addressing critical security vulnerabilities that threaten spectrum efficiency and network integrity. Through comprehensive evaluation across multiple clustering algorithms including K-means, Spectral, and Agglomerative clustering, enhanced with K-Nearest Neighbors and statistical mean-based classification techniques, this work demonstrates significant improvements over existing detection methodologies. The proposed framework exhibits superior performance across varying attack intensities and spatial configurations, providing robust protection against sophisticated emulation attacks while maintaining low false detection rates. The distance-dependent analysis reveals important insights into attack vector characteristics, enabling adaptive security strategies for different deployment scenarios. The comprehensive performance evaluation establishes clear deployment guidelines for cognitive radio system designers, offering practical solutions that balance detection accuracy with computational efficiency. These contributions advance the field of cognitive radio security by providing a scalable, reliable detection framework that significantly enhances the resilience of dynamic spectrum access systems against primary user emulation attacks, ultimately supporting the continued evolution of intelligent wireless communication networks.
