\chapter{Literature Review}
\label{Chapter3}
\lhead{Chapter 3. \emph{Literature Review}}

% {\LARGE\textbf{Literature Review Table: \ref{tab:LR_view1}
% Table: \ref{tab:LR_view2}}}

\begin{table}[]
    \centering
\begin{tabular}{|p{0.5cm}|p{2.5cm}|p{1.0cm}|p{3.2cm}|p{2.9cm}|p{2.8cm}|}
\hline
\textbf{SL. No.} & \textbf{Author} & \textbf{Date} & \textbf{Title} & \textbf{Contribution} & \textbf{Methodology} \\

\hline
1 & Ian F. Akyildiz, Won-Yeol Lee, Mehmet C. Vuran *, Shantidev Mohan & 2 May 2006 & NeXt generation/dynamic spectrum access/cognitive
radio wireless networks: A survey & xG networks (cognitive radio-based) as a solution to inefficient spectrum usage and analyzes their architecture, key functions, and impact on network protocols & survey-based analysis is used to explore functionalities, challenges, and cross-layer issues in xG networks.\\
\hline
2 & Ruiliang Chen, Jung-Min Park, and Jeffrey H. Reed & 31 Jan, 2008 & Defense against Primary User Emulation Attacks in
Cognitive Radio Networks & Primary User Emulation (PUE) attack as a serious threat to spectrum sensing in cognitive radio networks and proposes a localization-based defense mechanism, LocDef, to detect such attacks & proposed LocDef scheme using non-interactive localization to verify transmitter identity \\
\hline
3 & Z. Jin, S. Anand and K.P. Subbalakshmi & 11 Aug, 2009 & Detecting Primary User Emulation Attacks in Dynamic Spectrum Access Networks & Proposed a non-location-based method to detect PUEA in cognitive radio networks & Used Fenton’s approximation and WSPRT, supported by simulations\\
\hline
4 & Zesheng Chen, Todor Cooklev, Chao Chen, and Carlos Pomalaza-R´aez & 02 Feb, 2010 & Modeling Primary User Emulation Attacks and
Defenses in Cognitive Radio Networks & advanced PUEA strategies and proposes an effective countermeasure using channel invariants & estimation and learning techniques for both attack and defense, demonstrating their impact through comparative analysis\\

\hline
5 & Yao Liu, Peng Ning and Huaiyu Dai & 08 Jul, 2010 & Authenticating Primary Users' Signals in Cognitive Radio Networks via Integrated Cryptographic and Wireless Link Signatures & primary user authentication method in cognitive radio networks using helper nodes that combine cryptographic and physical-layer link signatures to detect malicious signal emulation & Training-free physical layer authentication technique leveraging helper nodes placed near primary users, conforming to FCC constraints and evaluated through theoretical design \\
\hline
6 & Bilal Naqvi, Imran Rashid, Fasisal Riaz and Baber Aslam & 10 Feb, 2014 & Primary User Emulation attack and their mitigation strategies: A survey &  Primary User Emulation (PUE) attacks in cognitive radio networks, analyzes existing mitigation techniques, and highlights unaddressed gaps requiring future solutions & survey-based approach that critically examines current PUE attack defenses and identifies their limitations to motivate further research.\\

\hline

\end{tabular}
    \caption{Literature Review}
    \label{tab:LR_view1}
\end{table}

\begin{table}[]
    \centering
\begin{tabular}{|p{0.5cm}|p{2.5cm}|p{1.1cm}|p{2.8cm}|p{2.8cm}|p{2.8cm}|}
\hline
\textbf{SL. No.} & \textbf{Author} & \textbf{Date} & \textbf{Title} & \textbf{Contribution} & \textbf{Methodology} \\
\hline
7 & Mohammad Javad Saber and Seyed Mohammad Sajad Sadough & 05 Jan, 2015 & Robust cooperative spectrum sensing in cognitive radio networks under multiple smart primary user emulation attacks &  cooperative spectrum sensing scheme to counter multiple smart Primary User Emulation Attacks (PUEAs) in cognitive radio networks by optimizing signal fusion to improve detection accuracy & Simulation-based evaluation of a fusion-center-driven sensing approach that maximizes the Cognitive Signal-to-Interference-plus-Noise Ratio (CSINR) for robust detection under coordinated PUEAs \\
\hline
8 & Dinu Mary Alias, Ragesh G. K & 15 Sep, 2016 & Cognitive Radio Networks: A Survey & Dynamic spectrum access improves spectrum efficiency & Centralized sharing, Distributed sharing \\
\hline
9 & D. L. Chaitanya and K. M. Chari & 04 May 2017 & Defense against PUEA and SSDF attacks in cognitive radio networks & Evaluate the performance of CRN. Also to mitigate the effect of SSDF attack & Hard Decision fusion rules are applied such as AND, OR and K out of N rules \\
\hline
10 & Khaled Mohammed Saifuddin et al. & 01 Sep, 2017 & Detection of Primary User Emulation Attack in Cognitive Radio Environment & Different methods to model communication channels and improve signal measurement accuracy & Filter and cyclostationary feature detection, spectrum decision \\
\hline
11 & Ishu Gupta, O. P. Sahu & 01 Feb, 2018 & An Overview of Primary User Emulation Attack in Cognitive Radio Networks & General overview of cognitive radio security issues with focus on PUEA & Energy detection, cyclostationary feature based detection \\
\hline

12 & Khatereh Akbari, Jamshid Abouei & 01 May, 2018 & Signal Classification for Detecting Primary User Emulation Attack in Centralized CRNs & Effective sequential scheme to identify PUE attack in CRNs & Bayesian nonparametric clustering approach based on DPMM \\
\hline
13 & N. Sureka and K. Gunaseelan & 11 Mar 2019 & Detection \& Defense against Primary User Emulation Attack in Dynamic Cognitive Radio Networks & Minimum interference to primary transmissions & Yardstick based Threshold Allocation (YTA) \\
\hline
14 & S. Arun \& G. Umamaheswari & 29 Apr 2019 & An Adaptive Learning-Based Attack Detection Technique for Mitigating Primary User Emulation in Cognitive Radio Networks & Proposes adaptive learning-based attack detection for PUEA, Enhances signal classification and SU communication rate. & Adaptive learning-based attack detection method, Cyclostationary feature analysis for signal classification \\
\hline
\end{tabular}
    \caption{Literature Review}
    \label{tab:LR_view1}
\end{table}


\begin{table}[]
    \centering
\begin{tabular}{|p{0.5cm}|p{2.5cm}|p{1.1cm}|p{2.8cm}|p{2.8cm}|p{2.8cm}|}
\hline
\textbf{SL. No.} & \textbf{Author} & \textbf{Date} & \textbf{Title} & \textbf{Contribution} & \textbf{Methodology} \\
\hline
15 & Wangjam Niranjan Singh, Ningrinla Marchang
and Amar Taggu & 25 Sept, 2019 & Mitigating SSDF attack using distance-based outlier approach in cognitive radio networks & Diving into two groups, non-outlier set and candidate set
 & Distance-based outlier approach \\
 \hline
 16 & Nikita Mishra, Sumit Srivastava and Shivendra Nath Sharan & 16 Jun, 2020 & Countermeasures for Primary User Emulation Attack: A Comprehensive Review & Research on PUEA in cognitive radio networks, offering insights and future strategies for secure and energy-efficient CR systems & survey and synthesis of existing literature, concluding with recommendations for next-generation CR security \\
\hline
17 & Avila Jayapalan, Prem Savarinathan, Jagathi Chenna Reddy & 07 Feb 2021 & Detection and Defense of PUEA in Cognitive Radio Network & PUEA detection using feature detection-based sensing with thresholds, Game model for strategic defense decisions against attackers. & Feature detection-based sensing with double threshold for PUEA detection, Game model for strategic defense decisions against attackers. 
\\
\hline
18 & Bishal Chhetry, Ningrinla Marchanga & 21 Jun, 2021 & Detection Of PUEA In CRNs Using One-Class Classification & One-class classification used for detecting PUEA & Isolation Forest, Support Vector Machine, MCD and LOF \\
\hline
19 & Grace Olaeru, Henry Ohize, Abubakar Saddiq Mohammed and Umar Suleiman Dauda & 24 Mar, 2022 & Optimal Detection Technique for Primary User Emulator in Cognitive Radio Network & Detects PUEA in CRNs using TDOA-based localization enhanced with optimization algorithms, identifying MPSO as the most effective & Simulation-based comparison of PSO, NBA, and MPSO using MATLAB and Monte Carlo trials, evaluated via MSE and CDF \\
\hline
20 & Amar Taggu, Ningrinla Marchang & 05 Jul, 2022 & A Density-based Clustering Approach to detect Colluding SSDF Attackers & Proposed DBSCAN-based technique for detecting Colluding SSDF attacks & Density-Based Spatial Clustering (DBSCAN) \\

\hline
21 & H. Thakkar and S. Goswami & 09 Aug 2023 & The Cluster-based Cognitive Radio Sensor Networks that are Wireless Aware of PUEA and SSDF Attacks & Resolve Spectrum Sensing
Data Falsification attack and Primary User Emulation attack & regression based spectrum
sensing data falsification attack detection technique and
mathematical SHA with Digital Signature \\
\hline
22 & C. Ambhika & 01 Apr 2024 & Discrimination of primary user emulation attack on cognitive radio networks using machine learning based spectrum sensing scheme & Detection and prevention of primary user emulation attack.
Improved sensing ability and energy efficiency in networks. & Time-Distance with signal Strength Evaluation (TDSE),
Extreme Machine Learning (EML) algorithm \\
\hline
\end{tabular}
    \caption{Literature Review}
    \label{tab:LR_view2}
\end{table}



This section presents a comprehensive analysis of the existing literature on Primary User Emulation Attack (PUEA) detection and mitigation techniques in Cognitive Radio Networks. The review is organized into major categories based on the methodological approaches employed by researchers.


Energy detection remains one of the fundamental approaches for spectrum sensing and PUEA detection in cognitive radio networks. This technique relies on measuring the energy of received signals and comparing them against predefined thresholds to determine the presence or absence of primary users.

Gupta and Sahu \cite{ref1} provided a comprehensive overview of primary user emulation attacks, emphasizing the effectiveness of energy detection methods combined with cyclostationary feature-based detection. Their work titled ``An Overview of Primary User Emulation Attack in Cognitive Radio Networks" highlighted that energy detection, while simple to implement, suffers from limitations in low Signal-to-Noise Ratio (SNR) environments and requires careful threshold selection to minimize false alarm rates. The authors comprehensively analyzed various detection techniques and established the foundation for understanding PUEA vulnerabilities in cognitive radio systems.
Chen et al. \cite{ref4} developed defense mechanisms against primary user emulation attacks in cognitive radio networks, focusing on authentication-based approaches. Their work ``Defense against primary user emulation attacks in cognitive radio networks" established fundamental defense strategies that complement energy detection methods by providing cryptographic verification of legitimate primary user signals.
Jin et al. \cite{ref6} proposed detection methods for primary user emulation attacks in dynamic spectrum access networks. Their research ``Detecting primary user emulation attacks in dynamic spectrum access networks" introduced novel approaches for identifying emulated signals through statistical analysis of signal characteristics and temporal patterns, providing a foundation for threshold-based detection systems.


The application of machine learning techniques has emerged as a promising approach for addressing the complexities of PUEA detection. These methods leverage pattern recognition and statistical learning to identify subtle differences between legitimate primary user signals and emulated signals.
Wang et al. \cite{ref9} proposed machine learning techniques for primary user emulation attack detection in cognitive radio in their research ``Primary user emulation attack detection in cognitive radio using machine learning techniques". Their approach leverages advanced algorithms to analyze spectrum sensing data and identify patterns indicative of emulation attacks, demonstrating superior performance in dynamic environments where traditional methods may fail.
Chhetry and Marchang \cite{ref23} explored one-class classification methods for PUEA detection in their comprehensive study ``Detection of PUEA in CRNs using one-class classification", employing multiple algorithms including Isolation Forest, Support Vector Machine (SVM), Minimum Covariance Determinant (MCD), and Local Outlier Factor (LOF). Their comparative study revealed that one-class classification is particularly effective when labeled data for attacks is scarce, as it only requires examples of normal behavior for training. The Isolation Forest algorithm showed superior performance in detecting novel attack patterns due to its ability to isolate anomalies through random partitioning.
Arun and Umamaheswari \cite{ref19} developed an adaptive learning-based attack detection technique for mitigating primary user emulation in cognitive radio networks. Their work ``An Adaptive Learning-Based Attack Detection Technique for Mitigating Primary User Emulation in Cognitive Radio Networks" demonstrates how machine learning can adapt to evolving attack strategies, providing robust detection capabilities that improve over time through continuous learning.


Cooperative spectrum sensing leverages the spatial diversity of multiple secondary users to improve detection accuracy and mitigate the effects of shadowing and fading. This approach is particularly effective against sophisticated PUEA strategies that may fool individual sensors.
Huang et al. \cite{ref8} focused on robust collaborative spectrum sensing in the presence of primary user emulation attacks in their work ``Robust collaborative spectrum sensing in the presence of primary user emulation attacks". Their methodology develops advanced fusion techniques that can maintain detection accuracy even when some participating nodes are compromised by attackers. The collaborative approach enhances overall network resilience by leveraging the spatial diversity of multiple sensing nodes.
Singh et al. \cite{ref14} proposed a distance-based outlier approach for mitigating SSDF attacks in their paper ``Mitigating SSDF Attack using Distance-based Outlier approach in Cognitive Radio Networks", which complements PUEA detection efforts. Their method divides secondary users into two groups: a non-outlier set consisting of trusted users and a candidate set containing potentially malicious users. The distance-based clustering algorithm identifies outliers by measuring the statistical distance between user reports and the expected behavior patterns.
Thakkar and Goswami \cite{ref12} addressed the challenges of wireless cognitive radio sensor networks by developing cluster-based approaches that are aware of both PUEA and SSDF attacks in their research ``The Cluster-based Cognitive Radio Sensor Networks that are Wireless Aware of PUEA and SSDF Attacks". Their methodology combines regression-based spectrum sensing data falsification attack detection with mathematical Secure Hash Algorithm (SHA) and digital signatures for authentication. This multi-layered security approach ensures both detection accuracy and data integrity in distributed sensing scenarios.


Clustering algorithms have gained significant attention for PUEA detection due to their ability to group similar signal characteristics and identify anomalous behaviors that may indicate attacks.
Taggu and Marchang \cite{ref21} proposed a density-based clustering approach using DBSCAN (Density-Based Spatial Clustering) to detect colluding SSDF attackers in their research ``A Density-based Clustering Approach to detect Colluding SSDF Attackers in Cognitive Radio Networks". Their method is particularly effective against coordinated attacks where multiple malicious users collaborate to deceive the network. The DBSCAN algorithm identifies clusters of normal behavior and flags data points that fall outside these clusters as potential attacks. The density-based approach is robust against noise and can identify clusters of arbitrary shapes, making it suitable for complex attack scenarios.
Luo et al. \cite{ref13} developed a novel clustering-based feature extraction method for PUEA detection in their work ``A novel clustering-based feature extraction method for PUEA detection". Their approach combines clustering algorithms with feature extraction techniques to identify distinctive patterns in spectrum sensing data that can effectively distinguish between legitimate and emulated signals.
The clustering-based approaches demonstrate particular strength in scenarios where attackers employ sophisticated strategies to mimic legitimate user behavior. By analyzing the clustering patterns of sensing reports over time, these methods can identify subtle deviations that may not be apparent through individual signal analysis.



Feature-based detection methods focus on analyzing specific characteristics of received signals to distinguish between legitimate and emulated transmissions. These approaches exploit the inherent differences in signal properties that are difficult for attackers to replicate perfectly.
Jin et al. \cite{ref15} developed detection methods for primary user emulation attacks in dynamic spectrum access networks in their paper ``Detecting primary user emulation attacks dynamic spectrum access networks". Their approach focuses on analyzing signal characteristics and temporal patterns to identify emulated signals, providing a foundation for feature-based detection systems that can adapt to various attack strategies.
Lafia et al. \cite{ref11} proposed a signal processing-based model for primary user emulation attacks detection in cognitive radio networks in their work ``Signal Processing-based Model for Primary User Emulation Attacks Detection in Cognitive Radio Networks". Their approach combines advanced signal processing techniques with feature extraction methods to create robust detection mechanisms that can identify subtle differences between legitimate and emulated signals.
The feature-based detection approaches often combine multiple signal characteristics such as power spectral density, modulation parameters, and temporal patterns to create comprehensive feature vectors that can effectively distinguish between legitimate and emulated signals. These methods demonstrate particular effectiveness in scenarios where attackers attempt to closely mimic legitimate primary user characteristics.



Authentication-based approaches represent a fundamental class of defense mechanisms against PUEA, focusing on verifying the legitimacy of primary user signals through cryptographic and signal-based authentication methods.
Liu et al. \cite{ref5} developed an integrated approach for ``Authenticating primary users' signals in cognitive radio networks via integrated cryptographic and wireless link signatures". Their method combines cryptographic techniques with wireless link characteristics to create a robust authentication framework that can verify the legitimacy of primary user transmissions, making it extremely difficult for attackers to successfully emulate authenticated signals.
Chen et al. \cite{ref17} presented analytical models for primary user emulation attacks and defenses in their work ``Modeling primary user emulation attacks and defenses in cognitive radio networks". Their research provides theoretical foundations for understanding the interaction between attackers and defense mechanisms, offering insights into optimal authentication strategies and their effectiveness against various attack scenarios.
Anand et al. \cite{ref16} developed ``An Analytical Model for Primary User Emulation Attacks in Cognitive Radio Networks" that provides mathematical frameworks for understanding attack strategies and defense mechanisms. Their analytical approach helps in designing more effective authentication protocols by predicting attacker behavior and system vulnerabilities.



Several foundational works have established the theoretical framework for understanding cognitive radio networks and PUEA threats, providing comprehensive surveys and establishing research directions.

Alias and Ragesh \cite{ref2} provided a comprehensive survey titled ``Cognitive Radio networks: A survey" that established the fundamental principles of cognitive radio networks. Their work demonstrated how dynamic spectrum access can be implemented through various architectural approaches, laying the groundwork for understanding both the opportunities and vulnerabilities in cognitive radio systems.
Akyildiz et al. \cite{ref3} presented a foundational survey on ``Next generation/dynamic spectrum access/cognitive radio wireless networks: A survey" that established the theoretical framework for cognitive radio technology. Their comprehensive analysis covers the evolution from traditional spectrum management to dynamic spectrum access, providing the conceptual foundation that underlies much of the subsequent research on security vulnerabilities including PUEA.
Naqvi et al. \cite{ref7} provided a comprehensive review of ``Primary User Emulation attack and their mitigation strategies: A survey" that systematically analyzed various PUEA attack vectors and corresponding defense mechanisms. Their survey established a taxonomy of attack types and mitigation approaches, serving as a critical reference for researchers developing new detection and defense strategies.
Parvin et al. \cite{ref25} conducted an extensive survey on ``Cognitive radio network security: A survey" that provides a broad perspective on security challenges in cognitive radio networks, including PUEA as well as other security threats. Their work establishes the broader security context within which PUEA detection and mitigation techniques operate.


Recent advances in deep learning have opened new possibilities for PUEA detection, offering sophisticated pattern recognition capabilities that can adapt to evolving attack strategies.
Zhao \cite{ref22} developed ``A reliable spectrum sensing method based on deep learning for primary user emulation attack detection in cognitive radio network". Their approach leverages deep neural networks to analyze complex patterns in spectrum sensing data, demonstrating superior performance in detecting sophisticated emulation attacks that may fool traditional detection methods. The deep learning model can automatically extract relevant features and adapt to new attack patterns through continuous learning.
Olaleru et al. \cite{ref20} proposed an ``Optimal Detection Technique for Primary User Emulator in Cognitive Radio Network" that incorporates advanced optimization techniques for improving detection accuracy. Their work focuses on developing optimal detection strategies that can maximize detection probability while minimizing false alarm rates.
Mishra et al. \cite{ref24} provided a comprehensive review of ``Countermeasures for primary user emulation attack: A comprehensive review" that systematically analyzes various defense mechanisms including emerging deep learning approaches. Their survey highlights the evolution from traditional signal processing methods to advanced machine learning and deep learning techniques for PUEA detection.



Despite the significant progress in PUEA detection techniques, several critical limitations persist across the reviewed literature that hinder the development of comprehensive and robust detection systems. Most studies concentrate on a single detection algorithm without comprehensive comparative analysis across different methodological approaches under varying conditions, as evidenced in the works of Gupta and Sahu \cite{ref1}, Jin et al. \cite{ref6}, and Chhetry and Marchang \cite{ref23}, which focus exclusively on their respective detection methods without systematic comparison with alternative approaches. There is insufficient exploration of how different algorithms perform across varying spatial configurations and attacker distributions, with studies like Chen et al. \cite{ref4} and Arun and Umamaheswari \cite{ref19} primarily evaluating their methods under limited spatial scenarios. Few studies systematically combine multiple detection techniques to leverage the strengths of different approaches, with notable exceptions being the works of Liu et al. \cite{ref5} and Thakkar and Goswami \cite{ref12}, though even these hybrid approaches remain limited in scope. Many proposed methods lack rigorous statistical validation of performance improvements, particularly in terms of statistical significance testing, as observed in the machine learning approaches by Wang et al. \cite{ref9} and the clustering methods by Taggu and Marchang \cite{ref21}, which rely primarily on simulation results without comprehensive statistical analysis. Additionally, most research focuses on simulation-based validation with limited consideration of practical implementation challenges and real-world constraints, as highlighted in the survey by Naqvi et al. \cite{ref7} and the comprehensive review by Mishra et al. \cite{ref24}, indicating a significant gap between theoretical proposals and practical deployment considerations. These gaps highlight the need for more comprehensive comparative studies and the development of robust hybrid detection mechanisms that can effectively address the evolving nature of PUEA threats in modern cognitive radio networks.



\section{\textbf{Countermeasure}}
Primary User Emulation Attacks (PUEA) pose significant threats to Cognitive Radio Networks (CRNs) by misleading secondary users about the availability of spectrum, necessitating the development of comprehensive countermeasures and mitigation strategies to ensure robust network security \cite{ref1,ref4,ref24}. The most effective approaches include cooperative spectrum sensing, which involves multiple secondary users collaboratively sensing the spectrum and sharing their observations to better distinguish between legitimate primary user signals and emulated signals, thereby enhancing detection accuracy and reducing false positives through collective sensing capabilities \cite{ref8}. Energy detection and thresholding methods measure the energy of received signals and compare them to predefined thresholds, adjusting detection parameters based on environmental conditions and historical data to minimize false alarms while maintaining effectiveness in low-noise environments. Feature-based detection techniques analyze specific signal characteristics such as modulation type and signal shape, comparing these features against known profiles of legitimate primary user signals to identify anomalies indicative of PUEA attacks \cite{ref11}. Location verification methods add an additional security layer by verifying the geographical location of signal sources using GPS data or other location-based technologies to ensure signals originate from expected locations where legitimate primary users operate. Advanced machine learning techniques utilize algorithms such as Support Vector Machines (SVM) and neural networks to detect PUEA by analyzing patterns in spectrum sensing data, offering the capability to adapt to new attack patterns and improve detection rates over time through continuous learning \cite{ref9,ref19,ref23}. Adaptive resource allocation strategies dynamically adjust power levels and communication parameters based on current network conditions and potential attack presence, enhancing overall network efficiency while reducing the likelihood of successful attacks \cite{ref3}. Cross-layer approaches integrate information from multiple network stack layers (physical, MAC, network) to provide a holistic view of network conditions and improve detection capabilities through data correlation from different layers. Trust-based mechanisms establish frameworks among secondary users to assess the reliability of their reports, giving more weight to users with accurate reporting histories while monitoring or excluding those suspected of malicious behavior, thereby encouraging cooperation and helping identify potential attackers \cite{ref5,ref8}. Mitigating Primary User Emulation Attacks in Cognitive Radio Networks requires this multifaceted approach that combines various detection and defense strategies, and by employing cooperative sensing, advanced detection techniques, and adaptive resource management, CRNs can enhance their resilience against PUEA and ensure efficient spectrum utilization, while ongoing research continues to explore innovative methods to further strengthen these defenses and improve the overall security of cognitive radio networks \cite{ref3,ref8,ref24,ref25}.